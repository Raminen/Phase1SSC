\section{Introduction}\label{introduction}

\section{Method and Results}\label{method-and-results}

\subsection{Find results from
Quansers}\label{find-results-from-quansers}

\begin{enumerate}

\item
  Step block (Start Time: 40s, Start Value: -2, End Value: -1 to 2) -
  This was used to automate the step input at a specific time for each
  elevator input test. This block replaced the existing constant value
  block which feeds into the elevator response.
\item
  Used range of values, 3 repeats. The elevator input was varied from
  and initial value of -2 to a range of values from -1 to 2 in steps of
  1.
\item
  The Workspace variables were saved for each repeat - this included
  pitch, elevator and travel data for each corresponding test.
\item
  Checked data for anomalies, Averaged repeats for valid results across
  4 different step inputs
\end{enumerate}

\subsection{Analyse Transfer Function}\label{analyse-transfer-function}

\begin{enumerate}

\item
  Isolated amplitude values for corresponding times after 40 seconds to
  80 seconds. 40 seconds refers to the step input time.
\item
  A second order sinusoidal decaying function was observed.
\item
  In order to estimate a transfer function - the following parameters
  were calculated: Natural frequency, Undamped natural frequency and
  damping ratio. Calculated using a 2nd order forced response function
  with step input
\item
  Natural Frequency
\item
  Undamped Natural Frequency
\item
  Damping Ratio
\end{enumerate}

\subsubsection{Second Order}\label{second-order}

The transfer function for a second order forced response should be as
follows: -

Correcting the Transfer function - Matlab's initial transfer function
step estimate showed phase and amplitude deviations from the
experimentally obtained result. For this reason the obtained transfer
function parameters were tweaked to achieve a closer fit. The parameters
that were changed (by small offsets to correct for deviations) in this
process include: damping ratio, undamped natural frequency

Due to the multiple steps taken for the Quanser, the response across the
different step inputs, the initial transfer functions for each were
calculated. These transfer functions were then averaged to give a mean
transfer function for the whole system, capturing some of the behaviour
of the system as step level varies. ( this is important when observing
the results). Finding the mean transfer function meant that only a
scaling factor k needs to be applied to get an approximate match for the
range of step inputs. To find this value, the max amplitude of the first
peak of all the step inputs was recorded. These values were then plotted
on a graph to find their correlation/ equation. Through trial and error
it was found that for step = 2, k = 0.26, this value was used to
translate correlation equation between points into scaling factors.
After applying this method, amplitudes for all the steps fitted more
closely. Finally damping ratio and undamped natural frequency where
tweaked to give the final fit, again by trial and error.

\subsubsection{First Order}\label{first-order}

\section{Results}\label{results}

\section{Observations and Analysis:}\label{observations-and-analysis}

\begin{itemize}
\tightlist
\item
  Step input of 2 was used as the datum which the scaling factors of the
  other steps was matched against. This results in the 2 step graph
  being a better fit than the others
\end{itemize}

\subsection{Potential Errors}\label{potential-errors}

\begin{itemize}
\tightlist
\item
  Accumulated error with time. Whilst the Quansers have error correcting
  -- this is not perfect. So longer the Quansers are run for the greater
  the accumulated error -- may get inconsistent results for longer time
  periods. This may explain the `drift' behaviour of the Quanser.
\item
  Whilst a step block used to increase accuracy of step input at the
  desired time in the Quanser run. The actual elevator input was not
  exactly at 40s. Accumulated lag in system and controller --
  computational latency (also effects initial condition which may
  explain phase difference).
\item
  Friction in hinge for Quanser
\item
  Wires caused partial decrease in deflection - `jamming'
\item
  There was not an immediate step response, thrust doesn't follow a step
  change
\item
  Noise in system (gyro noise, wind resistance)
\item
  Sensor sampling was discrete not continuous
\item
  As we sampled from -2 elevator the sinusoidal response was not
  completely damped after 40s -- hence the step input at 40s `added'
  onto the slight `upward' trailing motion
\item
  Transfer function is N dependent (only two curve peak values were
  taken into account)
\end{itemize}
